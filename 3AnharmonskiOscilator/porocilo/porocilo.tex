\documentclass{porocilo}
\institutionlogo{~/Pictures/UlFmf_logo.pdf}
\subjectname{Matematično-fizikalni praktikum}
\projecttitle{3. Anharmonski Oscilator}
\authorname{Tilen Šket, 28221057}
\instructions{Z diagonalizacijo poišči nekaj najnižjih lastnih
    vrednosti in lastnih valovnih funkcij za moteno Hamiltonko
    $H = \frac{1}{2} \left( p^2 + q^2 \right) + \lambda q^4$
    ob vrednostih parametra $0\le\lambda\le 1$.  Rešujemo torej
    matrični problem lastnih vrednosti
    \begin{equation*}
        H | n \rangle = E_n | n \rangle \>.
    \end{equation*}
}
\begin{document}
\maketitle

\section{Uvod}
\subsection{Fizikalni uvod}
Pri tej nalogi se bomo lotili reševanja problem anharmonskega oscilatorja. To je kvantno mehanski sistem s potencialom oblike
\begin{equation*}
    V = \frac{1}{2} q^2 + \lambda q^4.
\end{equation*}
Zaradi enostavnejšega zapisa bomo energije merili v enotah $\hbar \omega$, gibalne količine v enotah $(\hbar m \omega)^{1/2}$ in dolžine v enotah $(\hbar/m\omega)^{1/2}$. Hamiltonian zapišemo kot
\begin{equation*}
    H = H_0 + \lambda q^4,
\end{equation*}
kjer je
\begin{equation*}
    H_0 = \frac{1}{2} p^2 + \frac{1}{2} q^2.
\end{equation*}
Matrične elemente $H_0$, lahko zapišemo
\begin{equation*}
    \langle i|H_0|j \rangle = {1\over 2} \sqrt{i+j+1}\,\, \delta_{|i-j|,1} \>.
\end{equation*}
Matrične elemente perturbacijske matrike pa lahko izračunamo po treh metodah. Ali izračunamo matrične elemente posplošene koordinate in nato dvakrat kvadriramo dobljeno matriko
\begin{equation*}
    \langle i | q | j \rangle = {1\over 2} \sqrt{i+j+1}\,\, \delta_{|i-j|,1} \>,
\end{equation*}
ali izračunamo kar kvadrate le teh in matriko le enkrat kvadriramo
\begin{equation*}
    \langle i|q^2|j\rangle = {1\over 2} \biggl[
    {\sqrt{j(j-1)}} \, \delta_{i,j-2}
    + {(2j+1)} \, \delta_{i,j}
    + {\sqrt{(j+1)(j+2)}} \, \delta_{i,j+2} \biggr],
\end{equation*}
ali pa direktno četrto potenco po formuli
\begin{eqnarray*}
    \langle i|q^4|j\rangle
    = {1\over 2^4}\sqrt{2^i \, i!\over 2^{j} \, j! } \, \biggl[ \,
        &\,& \delta_{i,j+4} + 4\left(2j+3\right) \delta_{i,j+2}
        + 12 \left(2j^2+2j+1\right) \, \delta_{i,j} \\[3pt]
    &+& 16j \left(2j^2-3j+1\right) \, \delta_{i,j-2}
    + 16j\left(j^3-6j^2+11j-6\right) \, \delta_{i,j-4} \biggr] \>.
\end{eqnarray*}

Dobljene matrike nato diagonaliziramo in poiščemo njihove lastne vrednosti, ki nam povejo energijske nivoje, in lastne valovne funkcije, ki so zapisane v bazi lastnih valovnih funkcij harmonskega oscilatorja.

\subsection{Diagonalizacija}

\newpage
\section{Anharmonski oscilator}
\subsection{Način izračuna perturbacijske matrike}
Našemu hamiltonianu, sem na tri različne načine, opisane v uvodu, dodal perturbacijski del. V tem delu naloge, sem se vprašal katerega izmed teh bom uporabil pri kasnejših izračunih. Opazil sem, da se dobljene matrike ne skladajo popolnoma ampak, da dobimo pri kvadriranju robne napake.

\insertfig[0.9]{Matrike.png}{Grafično prikazane vrednosti perturbacijske matrike izračunane na tri različne metode. Iz slik se vidijo tudi robne napake.}{matrike}

Razlika med metodami je bila tudi v največji velikosti matrik, ki sem jih lahko še izračunal. Pri metodi, kjer sem direktno izračunal vrednost $H^4$, sem opazil zgornjo mejo pri okoli dimenziji 170, kar je povzročilo, da sem se odločil od tukaj dalje uporabljati metodo, kjer matriko kvadriram enkrat, torej računam $\langle i|q^2|j\rangle$.

\subsection{Diagonalizacija}
Pri tem delu naloge, sem dobljene matrike z različnimi implementacijami algoritmov diagonaliziral in poiskal lastne vrednosti ter vektorje. Različne implementacije sem primerjal po hitrosti diagonalizacije 1000 dimenzionalne matrike. Rezultati so zapisani v tabeli~\ref{tab:times}.

\begin{table}
    \centering
    \caption{\label{tab:times} Tabela različnih implementacij diagonalizacije matrike dimenzije 1000.}
    \begin{tabular}{c c}
        Implementacija & Čas [s]    \\
        numpy          & \num{0.31} \\
        scipy          & \num{0.25} \\
        Hausholder     & \num{89}   \\
        QR             & \num{68}   \\
    \end{tabular}
\end{table}

Od tukaj dalje sem uporabljal implementacijo iz knjižnjice numpy, saj sem je najbolj navajen in je bila primerljivo hitra kot tista iz knjižnjice scipy.

\subsection{Lastne vrednosti}
Matriko, ki sem jo ustvaril, sem sedaj diagonaliziral in poiskal njene lastne vrednosti s pomočjo implementacije tega v knjižnjici numpy. Zanimalo me je prvih nekaj lastnih vrednosti in njihova odvisnost od parametra $\lambda$. Podatke sem nato prikazal na~\ref{fig:eigen}

\insertfig{LastneVrednosti.png}{Odvisnost energije nivoja od zaporednega števila. Opazimo tudi, da pri $\lambda = 0$ rezultat sovpada z neperturbiranim.}{eigen}

Izračunal sem tudi odvisnost nekaterih prvih lastnih vrednosti od velikosti matrike uporabljene za izračun le teh. Dobra opomba je, da moramo za n-to lastno vrednost imeti matriko dimenzije vsaj n. Podatki so prikazani na sliki~\ref{fig:dimMat}

\insertfig{dimMatrike.png}{Odvisnost prvih nekaj lastnih vrednosti od števila dimenzij matrike.}{dimMat}

\subsection{Lastne valovne funkcije}
Poglejmo še, kako te lastne valovne funkcije izgledajo. Prikazane so na sliki~\ref{fig:vf}. Te sem dobil tako, da so linearne kombinacije nekih Hermitovih polinomov, v katerih bazi so bile po diagonalizaciji zapisane.

\insertfig{prikazVF.png}{Slika prvih štirih valovnih funkcij. Vsaka izmed teh je neka linearna kombinacija Hermitovih polinomov.}{vf}

\newpage
\section{Zaključek}
To nalogo oddajam v četrtek niti več zelo zgodaj zjutraj. Med že oddanimi mislim, da sem jo naredil najslabše, saj sem si vzel veliko premalo časa, ker sem jo začel delati šele v sredo pozno popoldne. Od takrat do sedaj pa se je ob praznovanju (še enega) rojstnega dneva v času oddaje naloge zgodilo veliko stvari, ki so zahtevale več moje pozornosti, kot sem to sprva načrtoval.

Še nekaj časovne statistike, ob pisanju tega zaključka sem na 6 urah skupnega dela, od česar pripadajo nekje 4 pisanju kode in preostali 2 pisanju tega poročila. To je časovno gledano dobro, vendar je tokrat zaradi časovne optimizacije preveč trpela kvaliteta, zato s tem nisem zadovoljen.

\end{document}
