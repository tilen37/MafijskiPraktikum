\newcommand{\institutionname}{Univerza v Ljubljani}
\newcommand{\projecttitle}{1. Izračun Airyjevih funkcij}
\newcommand{\authorname}{Tilen Šket, 28221057}
\newcommand{\instructions}{\textit{Z uporabo kombinacije Maclaurinove vrste in asimptotskega razvoja poišči čim učinkovitejši
        postopek za izračun vrednosti Airyjevih funkcij Ai in Bi na vsej realni osi z absolutno napako, manjšo
        od $10^{-10}$. Enako naredi tudi z relativno napako in ugotovi, ali je tudi pri le-tej dosegljiva natančnost,
        manjša od $10^{-10}$. Pri oceni napak si pomagaj s programi, ki znajo računati s poljubno natančnostjo,
        na primer z Mathematico in/ali paketi mpmath in decimal v programskem jeziku Python.}}
\newcommand{\subjectname}{Matematično-fizikalni praktikum}
\newcommand{\institutionlogo}{UlFmf_logo.pdf}

\begin{titlepage}
    \centering

    \includegraphics[width=0.7\textwidth]{\institutionlogo}

    \vspace{1.0cm}

    \Large
    \textbf{\subjectname}

    \vspace{1.0cm}

    \LARGE
    \textbf{\projecttitle}

    \vspace{1.0cm}

    \Large
    \textbf{\authorname}

    % \vspace{1.0cm}
    \vfill

    \large
    \textbf{Navodila:}

    \vspace{0.5cm}

    \large
    \instructions%

    \vspace{0.5cm}
    \vfill

    \large
    \textbf{\today}

\end{titlepage}