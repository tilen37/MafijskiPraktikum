\newcommand{\institutionname}{Univerza v Ljubljani}
\newcommand{\projecttitle}{2. Naključni Sprehodi}
\newcommand{\authorname}{Tilen Šket, 28221057}
\newcommand{\instructions}{\textit{
        Napravi računalniško simulacijo dvorazsežne naključne hoje za polete in sprehode. Začni vedno v izhodišču $(x = y = 0)$, nato pa določi naslednjo lego tako, da naključno
        izbereš smer koraka in statistično neodvisno od te izbire še njegovo dolžino, torej
        $p(l) \propto l^{-\mu}$.
        V vsakem primeru nariši nekaj značilnih slik sprehodov za 10, 100, 1000
        in 10 000 korakov. Iz velikega števila sprehodov z velikim številom korakov nato poskusi določiti eksponent $\gamma$ za nekaj izbranih parametrov $\mu$ oziroma funkcij $f(x)$ v posameznih primerih ter presodi, za kakšno vrsto difuzije gre.
    }}
\newcommand{\subjectname}{Matematično-fizikalni praktikum}
\newcommand{\institutionlogo}{UlFmf_logo.pdf}

\begin{titlepage}
    \centering

    \includegraphics[width=0.7\textwidth]{\institutionlogo}

    \vspace{1.0cm}

    \Large
    \textbf{\subjectname}

    \vspace{1.0cm}

    \LARGE
    \textbf{\projecttitle}

    \vspace{1.0cm}

    \Large
    \textbf{\authorname}

    % \vspace{1.0cm}
    \vfill

    \large
    \textbf{Navodila:}

    \vspace{0.5cm}

    \large
    \instructions%

    \vspace{0.5cm}
    \vfill

    \large
    \textbf{\today}

\end{titlepage}
